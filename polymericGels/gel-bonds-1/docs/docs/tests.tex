\documentclass[../main.tex]{subfiles}

\begin{document}

\marginnote{
\begin{itemize}
    \item etotal = pe + ke
    \item ecouple = commulative energy change due to thermo/baro 
    \item econserve = pe + ke + ecouple
    \item ebond = bond energy
    \item eangle = angle energy
    \item emol = ebond + eangle + edihed + eimp
\end{itemize}

The econserve keyword is the sum of the potential and kinetic energy of the system as well as the energy that has been transferred by thermostatting or barostatting to their coupling reservoirs – that is, econserve = pe + ke + ecouple. Ideally, for a simulation in the NVT, NPH, or NPT ensembles, the econserve quantity should remain constant over time even though etotal may change.
}

\subsection{Directory: 0.050.30.052500-2026-01-16-102953}

La simulación tardó, aproximadamente, 5 hrs.
Fueron \num{2500} partículas y \num{8.5d6} iteraciones.
Se asginó $\mathrm{damp}=1$ para ver que onda.
Fracción de empaquetamiento y concentración de crosslinkers son irrelevantes por el momento.

\begin{figure}[h]
    \centering
    \includegraphics[width=0.9\textwidth]{../Figures/0.050.30.052500-2026-01-16-102953/finalConfigAssembly.png}
    \caption{Configuración final del assembly}%\label{fig:strain-vs-normstress}
\end{figure}

El error proviene del archivo \verb|swapMech.3b|.
Estaba declarado de la siguiente forma:

\begin{code}
PA PA PA 0.6 swapMechTab1_w0.75.table SEC1 linear 100
PB PA PA 0.6 swapMechTab1_w0.75.table SEC1 linear 100
PA PB PA 0.6 swapMechTab1_w0.75.table SEC1 linear 100
PB PB PA 0.6 swapMechTab1_w0.75.table SEC1 linear 100
PA PA PB 0.6 swapMechTab1_w0.75.table SEC1 linear 100
PB PA PB 0.6 swapMechTab1_w0.75.table SEC1 linear 100
PA PB PB 0.6 swapMechTab1_w0.75.table SEC1 linear 100
PB PB PB 0.6 swapMechTab1_w0.75.table SEC1 linear 100
\end{code}

La diferencia con el archivo \verb|swapMechTab2_w0.75.table| es la evaluación de los potenciales.
En el archivo 1 tiene más elementos evaluados que en el segundo archivo.
Esto por la forma que tiene LAMMPS para evaluar potenciales \emph{numéricos}\footnote{%The tabulation is done on a three-dimensional grid of the two distances\ldots
\ldots There are two different cases. 
If element 2 and element 3 are of the same type (e.g. SiCC),\ldots %the distance is varied in “N” steps from rmin to rmax and the distance  is varied from  to rmax. \ldots
If element 2 and element 3 are not of the same type (e.g. SiCSi),\ldots %there is no additional symmetry and the distance  is also varied from rmin to rmax in “N” steps. 
%The angle  is always varied in “2N” steps from (0.0 + 180.0/(4N)) to (180.0 - 180.0/(4N)). 
Therefore, the total number of table entries is “M = N * N * (N+1)” for the symmetric (element 2 and element 3 are of the same type) and “M = 2 * N * N * N” for the general case (element 2 and element 3 are not of the same type).}.

\subsection{Directory: 0.050.30.05500-2026-01-20-111704}

\begin{marginfigure}
    \includegraphics[width=\linewidth]{../Figures/0.050.30.05500-2026-01-20-111704/temp-0.050.30.05500-2026-01-20-111704.png}
    \caption{Temperature during assembly.}
\end{marginfigure}

\begin{marginfigure}
    \includegraphics[width=\linewidth]{../Figures/0.050.30.05500-2026-01-20-111704/eSys-0.050.30.05500-2026-01-20-111704.png}
    \caption{Econserve and ecouple.}
\end{marginfigure}

\begin{marginfigure}
    \includegraphics[width=\linewidth]{../Figures/0.050.30.05500-2026-01-20-111704/eB-0.050.30.05500-2026-01-20-111704.png}
    \caption{Bonds de la partícula patchy}
\end{marginfigure}

\begin{marginfigure}
    \includegraphics[width=\linewidth]{../Figures/0.050.30.05500-2026-01-20-111704/epk-0.050.30.05500-2026-01-20-111704.png}
    \caption{ep y ek}
\end{marginfigure}

Main fix:
\begin{code}
PA PA PA 0.6 swapMechTab2_w0.75.table SEC1 linear 100
PB PA PA 0.6 swapMechTab2_w0.75.table SEC1 linear 100
PA PB PA 0.6 swapMechTab1_w0.75.table SEC1 linear 100
PB PB PA 0.6 swapMechTab1_w0.75.table SEC1 linear 100
PA PA PB 0.6 swapMechTab1_w0.75.table SEC1 linear 100
PB PA PB 0.6 swapMechTab1_w0.75.table SEC1 linear 100
PA PB PB 0.6 swapMechTab2_w0.75.table SEC1 linear 100
PB PB PB 0.6 swapMechTab2_w0.75.table SEC1 linear 100
\end{code}

Ahora, el problema principal es la aglomeración de interación de más de dos patches, se tiene que $w=0.75$,

\begin{figure}[h]
    \centering
    \includegraphics[width=0.9\textwidth]{../Figures/0.050.30.05500-2026-01-20-111704/triple.png}
\end{figure}


\newpage

\subsection{Directory: 0.050.30.05500-2026-01-20-121005}

\begin{marginfigure}
    \includegraphics[width=\linewidth]{../Figures/0.050.30.05500-2026-01-20-121005/temp-0.050.30.05500-2026-01-20-121005.png}
    \caption{Temperature during assembly.}
\end{marginfigure}

\begin{marginfigure}
    \includegraphics[width=\linewidth]{../Figures/0.050.30.05500-2026-01-20-121005/eSys-0.050.30.05500-2026-01-20-121005.png}
    \caption{Econserve and ecouple.}
\end{marginfigure}

\begin{marginfigure}
    \includegraphics[width=\linewidth]{../Figures/0.050.30.05500-2026-01-20-121005/eB-0.050.30.05500-2026-01-20-121005.png}
    \caption{Bonds de la partícula patchy}
\end{marginfigure}

\begin{marginfigure}
    \includegraphics[width=\linewidth]{../Figures/0.050.30.05500-2026-01-20-121005/epk-0.050.30.05500-2026-01-20-121005.png}
    \caption{ep y ek}
\end{marginfigure}

En esta simulación se cambió el parámetro $w$ a $w=1$,
\begin{figure}[h]
    \centering
    \includegraphics[width=0.9\textwidth]{../Figures/0.050.30.05500-2026-01-20-121005/triple.png}
\end{figure}

Parece que incrementar el parámetro $w$, incrementa la cantidad de interacciones entre patches.
Posiblemente sea error de signo en la evaluación de la fuerza o potencial, porque debería de pasar lo contrario, por la definición del potencial.

\newpage

\subsection{Directory: 0.050.30.05500-2026-01-20-135923}

\begin{marginfigure}
    \includegraphics[width=\linewidth]{../Figures/0.050.30.05500-2026-01-20-135923/temp-0.050.30.05500-2026-01-20-135923.png}
    \caption{Temperature during assembly.}
\end{marginfigure}

\begin{marginfigure}
    \includegraphics[width=\linewidth]{../Figures/0.050.30.05500-2026-01-20-135923/eSys-0.050.30.05500-2026-01-20-135923.png}
    \caption{Econserve and ecouple.}
\end{marginfigure}

\begin{marginfigure}
    \includegraphics[width=\linewidth]{../Figures/0.050.30.05500-2026-01-20-135923/eB-0.050.30.05500-2026-01-20-135923.png}
    \caption{Bonds de la partícula patchy}
\end{marginfigure}

\begin{marginfigure}
    \includegraphics[width=\linewidth]{../Figures/0.050.30.05500-2026-01-20-135923/epk-0.050.30.05500-2026-01-20-135923.png}
    \caption{ep y ek}
\end{marginfigure}

$w=2$
Para reducir el tiempo de espera, la cantidad de iteraciones se disminuyeron a \num{5.5d6}.
Se tardó 36 minutos.

\begin{figure}[h]
    \centering
    \includegraphics[width=0.9\textwidth]{../Figures/0.050.30.05500-2026-01-20-135923/triple.png}
\end{figure}

\newpage

\subsection{Directory: 0.050.30.05500-2026-01-20-143651}

\begin{marginfigure}
    \includegraphics[width=\linewidth]{../Figures/0.050.30.05500-2026-01-20-143651/temp-0.050.30.05500-2026-01-20-143651.png}
    \caption{Temperature during assembly.}
\end{marginfigure}

\begin{marginfigure}
    \includegraphics[width=\linewidth]{../Figures/0.050.30.05500-2026-01-20-143651/eSys-0.050.30.05500-2026-01-20-143651.png}
    \caption{Econserve and ecouple.}
\end{marginfigure}

\begin{marginfigure}
    \includegraphics[width=\linewidth]{../Figures/0.050.30.05500-2026-01-20-143651/eB-0.050.30.05500-2026-01-20-143651.png}
    \caption{Bonds de la partícula patchy}
\end{marginfigure}

\begin{marginfigure}
    \includegraphics[width=\linewidth]{../Figures/0.050.30.05500-2026-01-20-143651/epk-0.050.30.05500-2026-01-20-143651.png}
    \caption{ep y ek}
\end{marginfigure}



Ahora se explora con $w=0.5$
35 minutos.

\begin{figure}[h]
    \centering
    \includegraphics[width=0.9\textwidth]{../Figures/0.050.30.05500-2026-01-20-143651/triple.png}
\end{figure}

Mi intuición ahora es que puede que tenga mal las diferencias finitas.
En las simulaciones que use para la Tesis de maestría se tenía el $\mathrm{damp}=0.1$ y ahora es de $\mathrm{damp}=1$.
Intuyo que también por eso no se está logrando estabilizar mucho esto.
También, viendo rápidamente la temperatura, está se relaja arriba de \num{0.05}, lo cuál no es esperado. 

Volveré a intentar, pero cambiaré la masa de las partículas patchy a $1$, a ver que onda.

\newpage

\subsection{Directory: 0.050.30.05500-2026-01-20-151755}

\begin{marginfigure}
    \includegraphics[width=\linewidth]{../Figures/0.050.30.05500-2026-01-20-151755/temp-0.050.30.05500-2026-01-20-151755.png}
    \caption{Temperature during assembly.}
\end{marginfigure}

\begin{marginfigure}
    \includegraphics[width=\linewidth]{../Figures/0.050.30.05500-2026-01-20-151755/eSys-0.050.30.05500-2026-01-20-151755.png}
    \caption{Econserve and ecouple.}
\end{marginfigure}

\begin{marginfigure}
    \includegraphics[width=\linewidth]{../Figures/0.050.30.05500-2026-01-20-151755/eB-0.050.30.05500-2026-01-20-151755.png}
    \caption{Bonds de la partícula patchy}
\end{marginfigure}

\begin{marginfigure}
    \includegraphics[width=\linewidth]{../Figures/0.050.30.05500-2026-01-20-151755/epk-0.050.30.05500-2026-01-20-151755.png}
    \caption{ep y ek}
\end{marginfigure}

Ahora se explora con $w=1$
Masa de partículas patchy de $0.1$ a $1$.
32 minutos.

\begin{figure}[h]
    \centering
    \includegraphics[width=0.9\textwidth]{../Figures/0.050.30.05500-2026-01-20-151755/triple.png}
\end{figure}

No se ´´arreglo'' la interacción de más de dos patches, pero parece ser que la temperatura si se estabilizó en el valor esperado, \num{0.05}.


\newpage

\subsection{Directory: 0.050.30.05500-2026-01-21-133721}

\begin{marginfigure}
    \includegraphics[width=\linewidth]{../Figures/0.050.30.05500-2026-01-21-133721/temp-0.050.30.05500-2026-01-21-133721.png}
    \caption{Temperature during assembly.}
\end{marginfigure}

\begin{marginfigure}
    \includegraphics[width=\linewidth]{../Figures/0.050.30.05500-2026-01-21-133721/eSys-0.050.30.05500-2026-01-21-133721.png}
    \caption{Econserve and ecouple.}
\end{marginfigure}

\begin{marginfigure}
    \includegraphics[width=\linewidth]{../Figures/0.050.30.05500-2026-01-21-133721/eB-0.050.30.05500-2026-01-21-133721.png}
    \caption{Bonds de la partícula patchy}
\end{marginfigure}

\begin{marginfigure}
    \includegraphics[width=\linewidth]{../Figures/0.050.30.05500-2026-01-21-133721/epk-0.050.30.05500-2026-01-21-133721.png}
    \caption{ep y ek}
\end{marginfigure}



\end{document}
