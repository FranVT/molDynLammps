\documentclass[../main.tex]{subfiles}

\begin{document}

\subsection{Directory: 0.050.30.052500-2026-01-16-102953}

La simulación tardó, aproximadamente, 5 hrs.
Fueron \num{2500} partículas y \num{8.5d6} iteraciones.
Se asginó $\mathrm{damp}=1$ para ver que onda.
Fracción de empaquetamiento y concentración de crosslinkers son irrelevantes por el momento.

\begin{figure}[h]
    \centering
    \includegraphics[width=0.9\textwidth]{../Figures/0.050.30.052500-2026-01-16-102953/finalConfigAssembly.png}
    \caption{Configuración final del assembly}%\label{fig:strain-vs-normstress}
\end{figure}

El error proviene del archivo \verb|swapMech.3b|.
Estaba declarado de la siguiente forma:

\begin{code}
PA PA PA 0.6 swapMechTab1_w0.75.table SEC1 linear 100
PB PA PA 0.6 swapMechTab1_w0.75.table SEC1 linear 100
PA PB PA 0.6 swapMechTab1_w0.75.table SEC1 linear 100
PB PB PA 0.6 swapMechTab1_w0.75.table SEC1 linear 100
PA PA PB 0.6 swapMechTab1_w0.75.table SEC1 linear 100
PB PA PB 0.6 swapMechTab1_w0.75.table SEC1 linear 100
PA PB PB 0.6 swapMechTab1_w0.75.table SEC1 linear 100
PB PB PB 0.6 swapMechTab1_w0.75.table SEC1 linear 100
\end{code}

La diferencia con el archivo \verb|swapMechTab2_w0.75.table| es la evaluación de los potenciales.
En el archivo 1 tiene más elementos evaluados que en el segundo archivo.
Esto por la forma que tiene LAMMPS para evaluar potenciales \emph{numéricos}\footnote{%The tabulation is done on a three-dimensional grid of the two distances\ldots
\ldots There are two different cases. 
If element 2 and element 3 are of the same type (e.g. SiCC),\ldots %the distance is varied in “N” steps from rmin to rmax and the distance  is varied from  to rmax. \ldots
If element 2 and element 3 are not of the same type (e.g. SiCSi),\ldots %there is no additional symmetry and the distance  is also varied from rmin to rmax in “N” steps. 
%The angle  is always varied in “2N” steps from (0.0 + 180.0/(4N)) to (180.0 - 180.0/(4N)). 
Therefore, the total number of table entries is “M = N * N * (N+1)” for the symmetric (element 2 and element 3 are of the same type) and “M = 2 * N * N * N” for the general case (element 2 and element 3 are not of the same type).}.

\subsection{Directory: 0.050.30.05500-2026-01-20-111704}

Main fix:
\begin{code}
PA PA PA 0.6 swapMechTab2_w0.75.table SEC1 linear 100
PB PA PA 0.6 swapMechTab2_w0.75.table SEC1 linear 100
PA PB PA 0.6 swapMechTab1_w0.75.table SEC1 linear 100
PB PB PA 0.6 swapMechTab1_w0.75.table SEC1 linear 100
PA PA PB 0.6 swapMechTab1_w0.75.table SEC1 linear 100
PB PA PB 0.6 swapMechTab1_w0.75.table SEC1 linear 100
PA PB PB 0.6 swapMechTab2_w0.75.table SEC1 linear 100
PB PB PB 0.6 swapMechTab2_w0.75.table SEC1 linear 100
\end{code}

Ahora, el problema principal es la aglomeración de interación de más de dos patches, se tiene que $w=0.75$,

\begin{figure}[h]
    \centering
    \includegraphics[width=0.9\textwidth]{../Figures/0.050.30.05500-2026-01-20-111704/triple.png}
    \caption{Configuración final del assembly}%\label{fig:strain-vs-normstress}
\end{figure}

\newpage

\subsection{Directory: 0.050.30.05500-2026-01-20-121005}

En esta simulación se cambió el parámetro $w$ a $w=1$,
\begin{figure}[h]
    \centering
    \includegraphics[width=0.9\textwidth]{../Figures/0.050.30.05500-2026-01-20-121005/triple.png}
    \caption{Configuración final del assembly}%\label{fig:strain-vs-normstress}
\end{figure}

Parece que incrementar el parámetro $w$, incrementa la cantidad de interacciones entre patches.
Posiblemente sea error de signo en la evaluación de la fuerza o potencial, porque debería de pasar lo contrario, por la definición del potencial.

\subsection{Directory: 0.050.30.05500-2026-01-20-135923}

$w=2$
Para reducir el tiempo de espera, la cantidad de iteraciones se disminuyeron a \num{5.5d6}.
Se tardó 36 minutos.

\begin{figure}[h]
    \centering
    \includegraphics[width=0.9\textwidth]{../Figures/0.050.30.05500-2026-01-20-135923/triple.png}
    \caption{Configuración final del assembly}%\label{fig:strain-vs-normstress}
\end{figure}




\subsection{Directory: 0.050.30.05500-2026-01-20-143651}
Ahora se explora con $w=0.5$


\end{document}
