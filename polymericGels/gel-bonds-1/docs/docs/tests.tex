\documentclass[../main.tex]{subfiles}

\begin{document}

\justifying

\marginnote{
\begin{itemize}
    \item etotal = pe + ke
    \item ecouple = commulative energy change due to thermo/baro 
    \item econserve = pe + ke + ecouple
    \item ebond = bond energy
    \item eangle = angle energy
    \item emol = ebond + eangle + edihed + eimp
\end{itemize}

The econserve keyword is the sum of the potential and kinetic energy of the system as well as the energy that has been transferred by thermostatting or barostatting to their coupling reservoirs – that is, econserve = pe + ke + ecouple. Ideally, for a simulation in the NVT, NPH, or NPT ensembles, the econserve quantity should remain constant over time even though etotal may change.
}

\subsection{Directory: 0.050.30.052500-2026-01-16-102953}

La simulación tardó, aproximadamente, 5 hrs.
Fueron \num{2500} partículas y \num{8.5d6} iteraciones.
Se asginó $\mathrm{damp}=1$ para ver que onda.
Fracción de empaquetamiento y concentración de crosslinkers son irrelevantes por el momento.

\begin{figure}[h]
    \centering
    \includegraphics[width=0.9\textwidth]{../Figures/0.050.30.052500-2026-01-16-102953/finalConfigAssembly.png}
    \caption{Configuración final del assembly}%\label{fig:strain-vs-normstress}
\end{figure}

El error proviene del archivo \verb|swapMech.3b|.
Estaba declarado de la siguiente forma:

\begin{code}
PA PA PA 0.6 swapMechTab1_w0.75.table SEC1 linear 100
PB PA PA 0.6 swapMechTab1_w0.75.table SEC1 linear 100
PA PB PA 0.6 swapMechTab1_w0.75.table SEC1 linear 100
PB PB PA 0.6 swapMechTab1_w0.75.table SEC1 linear 100
PA PA PB 0.6 swapMechTab1_w0.75.table SEC1 linear 100
PB PA PB 0.6 swapMechTab1_w0.75.table SEC1 linear 100
PA PB PB 0.6 swapMechTab1_w0.75.table SEC1 linear 100
PB PB PB 0.6 swapMechTab1_w0.75.table SEC1 linear 100
\end{code}

La diferencia con el archivo \verb|swapMechTab2_w0.75.table| es la evaluación de los potenciales.
En el archivo 1 tiene más elementos evaluados que en el segundo archivo.
Esto por la forma que tiene LAMMPS para evaluar potenciales \emph{numéricos}\footnote{%The tabulation is done on a three-dimensional grid of the two distances\ldots
\ldots There are two different cases. 
If element 2 and element 3 are of the same type (e.g. SiCC),\ldots %the distance is varied in “N” steps from rmin to rmax and the distance  is varied from  to rmax. \ldots
If element 2 and element 3 are not of the same type (e.g. SiCSi),\ldots %there is no additional symmetry and the distance  is also varied from rmin to rmax in “N” steps. 
%The angle  is always varied in “2N” steps from (0.0 + 180.0/(4N)) to (180.0 - 180.0/(4N)). 
Therefore, the total number of table entries is “M = N * N * (N+1)” for the symmetric (element 2 and element 3 are of the same type) and “M = 2 * N * N * N” for the general case (element 2 and element 3 are not of the same type).}.

\subsection{Directory: 0.050.30.05500-2026-01-20-111704}

\begin{marginfigure}
    \includegraphics[width=\linewidth]{../Figures/0.050.30.05500-2026-01-20-111704/temp-0.050.30.05500-2026-01-20-111704.png}
    \caption{Temperature during assembly.}
\end{marginfigure}

\begin{marginfigure}
    \includegraphics[width=\linewidth]{../Figures/0.050.30.05500-2026-01-20-111704/eSys-0.050.30.05500-2026-01-20-111704.png}
    \caption{Econserve and ecouple.}
\end{marginfigure}

\begin{marginfigure}
    \includegraphics[width=\linewidth]{../Figures/0.050.30.05500-2026-01-20-111704/eB-0.050.30.05500-2026-01-20-111704.png}
    \caption{Bonds de la partícula patchy}
\end{marginfigure}

\begin{marginfigure}
    \includegraphics[width=\linewidth]{../Figures/0.050.30.05500-2026-01-20-111704/epk-0.050.30.05500-2026-01-20-111704.png}
    \caption{ep y ek}
\end{marginfigure}

Main fix:
\begin{code}
PA PA PA 0.6 swapMechTab2_w0.75.table SEC1 linear 100
PB PA PA 0.6 swapMechTab2_w0.75.table SEC1 linear 100
PA PB PA 0.6 swapMechTab1_w0.75.table SEC1 linear 100
PB PB PA 0.6 swapMechTab1_w0.75.table SEC1 linear 100
PA PA PB 0.6 swapMechTab1_w0.75.table SEC1 linear 100
PB PA PB 0.6 swapMechTab1_w0.75.table SEC1 linear 100
PA PB PB 0.6 swapMechTab2_w0.75.table SEC1 linear 100
PB PB PB 0.6 swapMechTab2_w0.75.table SEC1 linear 100
\end{code}

Ahora, el problema principal es la aglomeración de interación de más de dos patches, se tiene que $w=0.75$,

\begin{figure}[h]
    \centering
    \includegraphics[width=0.9\textwidth]{../Figures/0.050.30.05500-2026-01-20-111704/triple.png}
\end{figure}


\newpage

\subsection{Directory: 0.050.30.05500-2026-01-20-121005}

\begin{marginfigure}
    \includegraphics[width=\linewidth]{../Figures/0.050.30.05500-2026-01-20-121005/temp-0.050.30.05500-2026-01-20-121005.png}
    \caption{Temperature during assembly.}
\end{marginfigure}

\begin{marginfigure}
    \includegraphics[width=\linewidth]{../Figures/0.050.30.05500-2026-01-20-121005/eSys-0.050.30.05500-2026-01-20-121005.png}
    \caption{Econserve and ecouple.}
\end{marginfigure}

\begin{marginfigure}
    \includegraphics[width=\linewidth]{../Figures/0.050.30.05500-2026-01-20-121005/eB-0.050.30.05500-2026-01-20-121005.png}
    \caption{Bonds de la partícula patchy}
\end{marginfigure}

\begin{marginfigure}
    \includegraphics[width=\linewidth]{../Figures/0.050.30.05500-2026-01-20-121005/epk-0.050.30.05500-2026-01-20-121005.png}
    \caption{ep y ek}
\end{marginfigure}

En esta simulación se cambió el parámetro $w$ a $w=1$,
\begin{figure}[h]
    \centering
    \includegraphics[width=0.9\textwidth]{../Figures/0.050.30.05500-2026-01-20-121005/triple.png}
\end{figure}

Parece que incrementar el parámetro $w$, incrementa la cantidad de interacciones entre patches.
Posiblemente sea error de signo en la evaluación de la fuerza o potencial, porque debería de pasar lo contrario, por la definición del potencial.

\newpage

\subsection{Directory: 0.050.30.05500-2026-01-20-135923}

\begin{marginfigure}
    \includegraphics[width=\linewidth]{../Figures/0.050.30.05500-2026-01-20-135923/temp-0.050.30.05500-2026-01-20-135923.png}
    \caption{Temperature during assembly.}
\end{marginfigure}

\begin{marginfigure}
    \includegraphics[width=\linewidth]{../Figures/0.050.30.05500-2026-01-20-135923/eSys-0.050.30.05500-2026-01-20-135923.png}
    \caption{Econserve and ecouple.}
\end{marginfigure}

\begin{marginfigure}
    \includegraphics[width=\linewidth]{../Figures/0.050.30.05500-2026-01-20-135923/eB-0.050.30.05500-2026-01-20-135923.png}
    \caption{Bonds de la partícula patchy}
\end{marginfigure}

\begin{marginfigure}
    \includegraphics[width=\linewidth]{../Figures/0.050.30.05500-2026-01-20-135923/epk-0.050.30.05500-2026-01-20-135923.png}
    \caption{ep y ek}
\end{marginfigure}

$w=2$
Para reducir el tiempo de espera, la cantidad de iteraciones se disminuyeron a \num{5.5d6}.
Se tardó 36 minutos.

\begin{figure}[h]
    \centering
    \includegraphics[width=0.9\textwidth]{../Figures/0.050.30.05500-2026-01-20-135923/triple.png}
\end{figure}

\newpage

\subsection{Directory: 0.050.30.05500-2026-01-20-143651}

\begin{marginfigure}
    \includegraphics[width=\linewidth]{../Figures/0.050.30.05500-2026-01-20-143651/temp-0.050.30.05500-2026-01-20-143651.png}
    \caption{Temperature during assembly.}
\end{marginfigure}

\begin{marginfigure}
    \includegraphics[width=\linewidth]{../Figures/0.050.30.05500-2026-01-20-143651/eSys-0.050.30.05500-2026-01-20-143651.png}
    \caption{Econserve and ecouple.}
\end{marginfigure}

\begin{marginfigure}
    \includegraphics[width=\linewidth]{../Figures/0.050.30.05500-2026-01-20-143651/eB-0.050.30.05500-2026-01-20-143651.png}
    \caption{Bonds de la partícula patchy}
\end{marginfigure}

\begin{marginfigure}
    \includegraphics[width=\linewidth]{../Figures/0.050.30.05500-2026-01-20-143651/epk-0.050.30.05500-2026-01-20-143651.png}
    \caption{ep y ek}
\end{marginfigure}



Ahora se explora con $w=0.5$
35 minutos.

\begin{figure}[h]
    \centering
    \includegraphics[width=0.9\textwidth]{../Figures/0.050.30.05500-2026-01-20-143651/triple.png}
\end{figure}

Mi intuición ahora es que puede que tenga mal las diferencias finitas.
En las simulaciones que use para la Tesis de maestría se tenía el $\mathrm{damp}=0.1$ y ahora es de $\mathrm{damp}=1$.
Intuyo que también por eso no se está logrando estabilizar mucho esto.
También, viendo rápidamente la temperatura, está se relaja arriba de \num{0.05}, lo cuál no es esperado. 

Volveré a intentar, pero cambiaré la masa de las partículas patchy a $1$, a ver que onda.

\begin{figure}[h]
    \centering
    \includegraphics[width=0.9\textwidth]{../Figures/0.050.30.05500-2026-01-20-143651/cluster-patch-patch.png}
\end{figure}


\newpage

\subsection{Directory: 0.050.30.05500-2026-01-20-151755}

\begin{marginfigure}
    \includegraphics[width=\linewidth]{../Figures/0.050.30.05500-2026-01-20-151755/temp-0.050.30.05500-2026-01-20-151755.png}
    \caption{Temperature during assembly.}
\end{marginfigure}

\begin{marginfigure}
    \includegraphics[width=\linewidth]{../Figures/0.050.30.05500-2026-01-20-151755/eSys-0.050.30.05500-2026-01-20-151755.png}
    \caption{Econserve and ecouple.}
\end{marginfigure}

\begin{marginfigure}
    \includegraphics[width=\linewidth]{../Figures/0.050.30.05500-2026-01-20-151755/eB-0.050.30.05500-2026-01-20-151755.png}
    \caption{Bonds de la partícula patchy}
\end{marginfigure}

\begin{marginfigure}
    \includegraphics[width=\linewidth]{../Figures/0.050.30.05500-2026-01-20-151755/epk-0.050.30.05500-2026-01-20-151755.png}
    \caption{ep y ek}
\end{marginfigure}

Ahora se explora con $w=1$
Masa de partículas patchy de $0.1$ a $1$.
32 minutos.

\begin{figure}[h]
    \centering
    \includegraphics[width=0.9\textwidth]{../Figures/0.050.30.05500-2026-01-20-151755/triple.png}
\end{figure}

No se ´´arreglo'' la interacción de más de dos patches, pero parece ser que la temperatura si se estabilizó en el valor esperado, \num{0.05}.

\begin{figure}[h]
    \centering
    \includegraphics[width=0.9\textwidth]{../Figures/0.050.30.05500-2026-01-20-151755/cluster-patch-patch.png}
\end{figure}



\newpage

\subsection{Directory: 0.050.30.05500-2026-01-21-133721}

\begin{marginfigure}
    \includegraphics[width=\linewidth]{../Figures/0.050.30.05500-2026-01-21-133721/temp-0.050.30.05500-2026-01-21-133721.png}
    \caption{Temperature during assembly.}
\end{marginfigure}

\begin{marginfigure}
    \includegraphics[width=\linewidth]{../Figures/0.050.30.05500-2026-01-21-133721/eSys-0.050.30.05500-2026-01-21-133721.png}
    \caption{Econserve and ecouple.}
\end{marginfigure}

\begin{marginfigure}
    \includegraphics[width=\linewidth]{../Figures/0.050.30.05500-2026-01-21-133721/ePair-0.050.30.05500-2026-01-21-133721.png}
    \caption{Energía de potenciales de interacción}
\end{marginfigure}

\begin{marginfigure}
    \includegraphics[width=\linewidth]{../Figures/0.050.30.05500-2026-01-21-133721/epk-0.050.30.05500-2026-01-21-133721.png}
    \caption{ep y ek}
\end{marginfigure}

Se supone recomienda que econserve se debe mantener constante a lo largo de las simulaciones, pero claramente empezó a tener un comportamiento exponencial y pienso que eso no es deseado.
En las simulaciones pasadas ese comportamiento no se podía apreciar porque en el commando \verb|fix lagevin| no estaba activada la opción \verb|tally|, que permité guardar la suma acumulada de la energía que se introduce al sistema através del baño térmico.
Además, el problema de las interacciones entre tres patches no se ´´arregló''
\begin{figure}[h]
    \centering
    \includegraphics[width=0.9\textwidth]{../Figures/0.050.30.05500-2026-01-21-133721/triple.png}
\end{figure}

\begin{figure}[h]
    \centering
    \includegraphics[width=0.9\textwidth]{../Figures/0.050.30.05500-2026-01-21-133721/cluster-patch-patch.png}
\end{figure}

\newpage

\subsection{Directory: 0.050.30.05500-2026-01-21-161335}

Puse la kewyword \verb|zero| a \emph{yes}, para forzar que la suma de la fuerza estocásita fuera cero, a ver que pasa con las energías.
Se tardó 35 mins.

\begin{marginfigure}
    \includegraphics[width=\linewidth]{../Figures/0.050.30.05500-2026-01-21-161335/temp-0.050.30.05500-2026-01-21-161335.png}
    \caption{Temperature during assembly.}
\end{marginfigure}

\begin{marginfigure}
    \includegraphics[width=\linewidth]{../Figures/0.050.30.05500-2026-01-21-161335/eSys-0.050.30.05500-2026-01-21-161335.png}
    \caption{Econserve and ecouple.}
\end{marginfigure}

\begin{marginfigure}
    \includegraphics[width=\linewidth]{../Figures/0.050.30.05500-2026-01-21-161335/ePair-0.050.30.05500-2026-01-21-161335.png}
    \caption{Energía de potenciales de interacción}
\end{marginfigure}

\begin{marginfigure}
    \includegraphics[width=\linewidth]{../Figures/0.050.30.05500-2026-01-21-161335/epk-0.050.30.05500-2026-01-21-161335.png}
    \caption{ep y ek}
\end{marginfigure}


\begin{figure}[h]
    \centering
    \includegraphics[width=0.9\textwidth]{../Figures/0.050.30.05500-2026-01-21-161335/cluster-patch-patch.png}
\end{figure}

\newpage 

\section{Directory: 0.050.30.05500-2026-01-22-154929}

\marginpar{
    \textbf{Parámetros \emph{importantes} de la simulación}
    \[\mathrm{damp}=0.5;\qquad w=1\]
}

Cambie el damp, para ver que onda.

\begin{marginfigure}
    \includegraphics[width=\linewidth]{../Figures/0.050.30.05500-2026-01-22-154929/temp-0.050.30.05500-2026-01-22-154929.png}
    \caption{Temperature during assembly.}
\end{marginfigure}

\begin{marginfigure}
    \includegraphics[width=\linewidth]{../Figures/0.050.30.05500-2026-01-22-154929/eSys-0.050.30.05500-2026-01-22-154929.png}
    \caption{Econserve and ecouple.}
\end{marginfigure}

\begin{marginfigure}
    \includegraphics[width=\linewidth]{../Figures/0.050.30.05500-2026-01-22-154929/ePair-0.050.30.05500-2026-01-22-154929.png}
    \caption{Energía de potenciales de interacción}
\end{marginfigure}

\begin{marginfigure}
    \includegraphics[width=\linewidth]{../Figures/0.050.30.05500-2026-01-22-154929/epk-0.050.30.05500-2026-01-22-154929.png}
    \caption{ep y ek}
\end{marginfigure}


\begin{figure}[h]
    \centering
    \includegraphics[width=0.9\textwidth]{../Figures/0.050.30.05500-2026-01-22-154929/cluster-patch-patch.png}
\end{figure}

Parece que no cambió mucho.


\newpage

\section{Acerca del potencial de tres cuerpos}

Mi intuición es que hay algo mal en la forma en como estoy evaluando el potencial de tres cuerpos.
Porque la energía que se registra debido al potential de swap es comprable con la energía que se registra debido al potencial de interacción.
Posiblemente estoy evaluando mal los dominios.

Revisando con calma la última versión que usé para crear las tablas del potencial de tres cuerpos me encontré que estaba mal evaluada.
En la documentación de lammps se espera que $f_{i1}=-f_{j1},~f_{i2}=-f_{k1},~f_{j2}=-f_{k2}$ y esas propiedades no se cumplieron para todas las evaluaciones.

Esto se puede aprecair facilmente en la declaración de la función que evalua las fuerzas debido al potencial de tres cuarpos:
\begin{code}
    th = deg2rad(th);
    r_jk = sqrt(r_ij^2+r_ik^2-2*r_ij*r_ik*cos(th));

    f_i=forceSwap(w,eps_ij,eps_ik,eps_jk,sig_p,r_ij,r_ik);
    f_j=forceSwap(w,eps_ij,eps_ik,eps_jk,sig_p,r_ij,r_jk);
    f_k=forceSwap(w,eps_ij,eps_ik,eps_jk,sig_p,r_ik,r_jk);

    f_i1=f_i;
    f_i2=f_i*cos(th);
   
    f_j1=f_j;
    f_j2=f_j*(1-cos(th));

    f_k1=f_k*cos(th);
    f_k2=f_k*(1-cos(th));
\end{code}

Cambié el código al siguiente:

\begin{code}
    th = deg2rad(th);
    r_jk = sqrt(r_ij^2+r_ik^2-2*r_ij*r_ik*cos(th));

    f_i=forceSwap(w,eps_ij,eps_ik,eps_jk,sig_p,r_ij,r_ik);
    f_j=forceSwap(w,eps_ij,eps_ik,eps_jk,sig_p,r_ij,r_jk);
    f_k=forceSwap(w,eps_ij,eps_ik,eps_jk,sig_p,r_ik,r_jk);

    f_i1=f_i;
    f_i2=f_i*cos(th);
   
    f_j1=-f_i1;
    f_j2=f_j*(1-cos(th));

    f_k1=-f_i2;
    f_k2=-f_j2;
\end{code}

\newpage

\section{Directory: 0.050.30.05500-2026-01-22-163251}

\begin{marginfigure}
    \includegraphics[width=\linewidth]{../Figures/0.050.30.05500-2026-01-22-163251/temp-0.050.30.05500-2026-01-22-163251.png}
    \caption{Temperature during assembly.}
\end{marginfigure}

\begin{marginfigure}
    \includegraphics[width=\linewidth]{../Figures/0.050.30.05500-2026-01-22-163251/eSys-0.050.30.05500-2026-01-22-163251.png}
    \caption{Econserve and ecouple.}
\end{marginfigure}

\begin{marginfigure}
    \includegraphics[width=\linewidth]{../Figures/0.050.30.05500-2026-01-22-163251/ePair-0.050.30.05500-2026-01-22-163251.png}
    \caption{Energía de potenciales de interacción}
\end{marginfigure}

\begin{marginfigure}
    \includegraphics[width=\linewidth]{../Figures/0.050.30.05500-2026-01-22-163251/epk-0.050.30.05500-2026-01-22-163251.png}
    \caption{ep y ek}
\end{marginfigure}



Cambié la evaluación de las tablas de la forma anterior.
Disminuí de \num{5d6} a \num{3d6} en el proceso isotérmico para que tardará menos las simulciones de prueba.
Fue hasta \num{3d6} pporque observando la energía potential de interacción, se relajaba a un valor más o menos constante al rededor de esas iteraciones.

Tardó 16 mins


\begin{figure}[h]
    \centering
    \includegraphics[width=0.9\textwidth]{../Figures/0.050.30.05500-2026-01-22-163251/cluster-patch-patch.png}
\end{figure}

Tiene vibras que mejoró, voy a poner \num{8d6}

\newpage

\section{Directory: 0.050.30.05500-2026-01-22-165623}

\begin{marginfigure}
    \includegraphics[width=\linewidth]{../Figures/0.050.30.05500-2026-01-22-165623/temp-0.050.30.05500-2026-01-22-165623.png}
    \caption{Temperature during assembly.}
\end{marginfigure}

\begin{marginfigure}
    \includegraphics[width=\linewidth]{../Figures/0.050.30.05500-2026-01-22-165623/eSys-0.050.30.05500-2026-01-22-165623.png}
    \caption{Econserve and ecouple.}
\end{marginfigure}

\begin{marginfigure}
    \includegraphics[width=\linewidth]{../Figures/0.050.30.05500-2026-01-22-165623/ePair-0.050.30.05500-2026-01-22-165623.png}
    \caption{Energía de potenciales de interacción}
\end{marginfigure}

\begin{marginfigure}
    \includegraphics[width=\linewidth]{../Figures/0.050.30.05500-2026-01-22-165623/epk-0.050.30.05500-2026-01-22-165623.png}
    \caption{ep y ek}
\end{marginfigure}

\begin{figure}[h]
    \centering
    \includegraphics[width=0.9\textwidth]{../Figures/0.050.30.05500-2026-01-22-165623/cluster-patch-patch.png}
\end{figure}

No entiendo porque aún hay una cantidad considerable de enlaces entres tres patches.

\newpage

\section{Acerca del potencial de tres cuerpos}

Cambié el código al siguiente:

\begin{code}
    th = deg2rad(th);
    r_jk = sqrt(r_ij^2+r_ik^2-2*r_ij*r_ik*cos(th));

    f_i=forceSwap(w,eps_ij,eps_ik,eps_jk,sig_p,r_ij,r_ik);
    f_j=forceSwap(w,eps_ij,eps_ik,eps_jk,sig_p,r_ij,r_jk);
    f_k=forceSwap(w,eps_ij,eps_ik,eps_jk,sig_p,r_ik,r_jk);

    f_i1=f_i;
    f_i2=f_i*cos(th);
   
    f_j1=-f_i1;
    f_j2=f_j*cos(th);

    f_k1=-f_i2;
    f_k2=-f_j2;
\end{code}

Aún no estoy muy seguro del porque no está funcionando.
Hice el cambio del factor escala de $1-\cos\theta\to\cos\theta$, porque el primer factor de escala es cuando estaba de terco usar un \emph{tercer vector propio} para el espacio dos dimensional.

\subsection{Acerca de la simetría}

El potencial de tres cuerpos está definido de la siguiente forma
\begin{gather*}
    U_\mathrm{swap}\qty(r_{ij},r_{ik}) = \alpha U_3(r_{ij})U_3(r_{ik}).
\end{gather*}
Considerando la relación entre potencial y fuerza en interacciones conservativas, se tienen las siguientes expresiones\footnote{
\begin{align*}
    \vec{F}_{i} &= -\nabla U_\mathrm{swap}\qty(r_{ij},r_{ik}) \\
            &= -\pdv{r} \alpha U_3(r_{ij})U_3(r_{ik})\hat{e}_{r} \\
            &= -\alpha\left[ U_3(r_{ik})\pdv{r}U_3(r_{ij}) + U_3(r_{ij})\pdv{r}U_3(r_{ik}) \right]\hat{e}_{r} 
\end{align*}
} para la fuerza de cada partículs:

\begin{align*}
    \vec{F}_{i} &= -\alpha\left[ U_3(r_{ik})\pdv{r}U_3(r_{ij}) + U_3(r_{ij})\pdv{r}U_3(r_{ik}) \right]\hat{e}_{r} \\
    \vec{F}_{j} &=  -\alpha\left[ U_3(r_{jk})\pdv{r}U_3(r_{ij}) + U_3(r_{ij})\pdv{r}U_3(r_{jk}) \right]\hat{e}_{r} \\
    \vec{F}_{k} &= -\alpha\left[ U_3(r_{jk})\pdv{r}U_3(r_{ik}) + U_3(r_{ik})\pdv{r}U_3(r_{jk}) \right]\hat{e}_{r}
\end{align*}

Por argumento intuitivo/diagrama las fuerzas no son recíprocas $f_{ij}\neq -f_{ji}$.
Porque se depende de la distancia $r_{jk}$ y $f_{ik}$.

Analizando el caso en que $\vec{F}_{i} = -\vec{F}_{j}$,
\begin{align*}
    \vec{F}_{i} &= \vec{F}_{j} \\
    -\alpha\left[ U_3(r_{ik})\pdv{r}U_3(r_{ij}) + U_3(r_{ij})\pdv{r}U_3(r_{ik}) \right]\hat{e}_{r}
                &=
    \alpha\left[ U_3(r_{jk})\pdv{r}U_3(r_{ij}) + U_3(r_{ij})\pdv{r}U_3(r_{jk}) \right]\hat{e}_{r} \\
    -U_3(r_{ik})\pdv{r}U_3(r_{ij}) - U_3(r_{ij})\pdv{r}U_3(r_{ik})
                &=
    U_3(r_{jk})\pdv{r}U_3(r_{ij}) + U_3(r_{ij})\pdv{r}U_3(r_{jk}) \\
\end{align*}



%    d/dr[U(r_ij)U(r_ik)] = U(r_ik)d/dr[U(r_ij)] + U(r_ij)d/dr[U(r_ik)]


\newpage 

\section{Directory: 0.050.30.05500-2026-01-23-150058}


\begin{figure}[h]
    \centering
    \includegraphics[width=0.9\textwidth]{../Figures/0.050.30.05500-2026-01-23-150058/cluster-patch-patch.png}
\end{figure}

No se arregló.

20 mins.

\newpage

\section{Directory: 0.050.30.05500-2026-01-23-152756}


Realicé el siguiente cambio en la declaración del potencial
\begin{code}
    th = deg2rad(th);
    r_jk = sqrt(r_ij^2+r_ik^2-2*r_ij*r_ik*cos(th));

    f_i=forceSwap(w,eps_ij,eps_ik,eps_jk,sig_p,r_ij,r_ik);
    f_j=forceSwap(w,eps_ij,eps_ik,eps_jk,sig_p,r_ij,r_jk);
    f_k=forceSwap(w,eps_ij,eps_ik,eps_jk,sig_p,r_ik,r_jk);

    f_i1=f_i;
    f_i2=f_i*cos(th);
   
    f_j1=f_j;
    f_j2=f_j*cos(th);

    f_k1=f_k;
    f_k2=f_k*cos(th);
\end{code}


\begin{figure}[h]
    \centering
    \includegraphics[width=0.9\textwidth]{../Figures/0.050.30.05500-2026-01-23-152756/cluster-patch-patch.png}
\end{figure}


\newpage

\section{Directory: 0.050.30.05500-2026-01-23-171941}

Hice algo que pensaba que estaba mal, pero después de rebotar planteamientos con DeepSeek me rendí y quise probar a ver que onda:
\begin{align*}
    f_{i1} &= -w\epsilon_{ij}\pdv{U_3\qty(r_{ij})}{r_{ij}}U_3\qty(r_{jk}) \\
    f_{i2} &= -w\epsilon_{ij}U_3\qty(r_{ij})\pdv{U_3\qty(r_{jk})}{r_{ik}}
\end{align*}

\begin{code}
function force3(w,eps_ij,eps_ik,eps_jk,sig_p,r_ij,r_ik,th)
    th = deg2rad(th);
    r_jk = sqrt(r_ij^2+r_ik^2-2*r_ij*r_ik*cos(th));

    f_i1=-w*eps_jk*DiffU3(eps_ij,eps_ij,sig_p,r_ik)*U3(eps_ik,eps_jk,sig_p,r_ik);
    f_i2=-w*eps_jk*U3(eps_ij,eps_ij,sig_p,r_ik)*DiffU3(eps_ik,eps_jk,sig_p,r_ik);
   
    f_j1=-f_i1;
    f_j2=-w*eps_jk*U3(eps_ij,eps_ij,sig_p,r_ik)*DiffU3(eps_ik,eps_jk,sig_p,r_jk);

    f_k1=-f_i2;
    f_k2=-f_j2;

    eng=SwapU(w,eps_ij,eps_ik,eps_jk,sig_p,r_ij,r_ik) + SwapU(w,eps_ij,eps_ik,eps_jk,sig_p,r_ij,r_jk) + SwapU(w,eps_ij,eps_ik,eps_jk,sig_p,r_ik,r_jk)
    eng=round(eng/3,digits=2^7)

    return (f_i1,f_i2,f_j1,f_j2,f_k1,f_k2,eng)
end
\end{code}

Va a estar mal, porque me equivoqué en la evaluación de $f_i1$.

\begin{figure}[h]
    \centering
    \includegraphics[width=0.9\textwidth]{../Figures/0.050.30.05500-2026-01-23-171941/cluster-patch-patch.png}
\end{figure}


\newpage

\section{Directory: 0.050.30.05500-2026-01-23-174419}

No entiendo.

\begin{figure}[h]
    \centering
    \includegraphics[width=0.9\textwidth]{../Figures/0.050.30.05500-2026-01-23-174419/cluster-patch-patch.png}
\end{figure}

\newpage

\section{Directory: 0.050.30.05500-2026-01-26-101526}

Cambié de signo, a ver que pasa.

\begin{figure}[h]
    \centering
    \includegraphics[width=0.9\textwidth]{../Figures/0.050.30.05500-2026-01-26-101526/cluster-patch-patch.png}
\end{figure}

\newpage

\subsection{Acerca de la tabulación del potencial de tres cuerpos}\label{subsec:tabpot}

Voy a asumir lo siguiente:

\begin{align*}
    f_{i1} &= -\pdv{r_{ij}}U_\mathrm{swap}(r_{ij},r_{ik}) = -\alpha \left[\pdv{U_3(r_{ij})}{r_{ij}}U_3(r_{ik})\right] \\
    f_{i2} &= -\pdv{r_{ik}}U_\mathrm{swap}(r_{ij},r_{ik}) = -\alpha \left[U_3(r_{ij})\pdv{U_3(r_{ik})}{r_{ik}}\right] \\
    f_{j1} &= -\pdv{r_{ij}}U_\mathrm{swap}(r_{ij},r_{jk}) = -\alpha \left[\pdv{U_3(r_{ij})}{r_{ij}}U_3(r_{jk})\right] \\
    f_{j2} &= -\pdv{r_{jk}}U_\mathrm{swap}(r_{ij},r_{jk}) = -\alpha \left[U_3(r_{ij})\pdv{U_3(r_{jk})}{r_{jk}}\right] \\
    f_{k1} &= -\pdv{r_{ik}}U_\mathrm{swap}(r_{ik},r_{jk}) = -\alpha \left[\pdv{U_3(r_{ik})}{r_{ik}}U_3(r_{jk})\right] \\
    f_{k2} &= -\pdv{r_{jk}}U_\mathrm{swap}(r_{ik},r_{jk}) = -\alpha \left[U_3(r_{ik})\pdv{U_3(r_{jk})}{r_{jk}}\right]
\end{align*}

Mientrás realizaba está tabla, me dí cuenta que estuve evaluando mal los potenciales y derivadas\footnote{A pesar de no tener la misma expressión algebraica, la evaluación estaba incorrecta.}.
Estás relaciones no siguen la relación $f_{i1} = -f_{j1}$, al menos que $r_{ik} = r_{jk}$.

La idea es que el sufijo $i,j,k$, infican partícula.
Los sufijos $1$ y $2$, lo intepreté como la contribución de la fuerza debido a la distancia $r_{lm}$ y la distancia $r_{ln}$.
Si el potencial también depende del ángulo, intuyo que hay que obtener las proyecciones a las distancias $r_{lm},r_{ln}$.

Aún no estoy seguro si hay que agregar el factor escala $r_{lm}$, ya que lammps hace la multiplicación con $\vec{r}_{lm}$.

\newpage

\section{Directory: 0.050.30.05500-2026-01-26-113021}

El cambio anterior no ayudo.

\begin{figure}[h]
    \centering
    \includegraphics[width=0.9\textwidth]{../Figures/0.050.30.05500-2026-01-26-113021/cluster-patch-patch.png}
\end{figure}

\newpage

\section{Directory: 0.050.30.05500-2026-01-26-120745}

Forcé el parámetro de reciprocidad 

\begin{figure}[h]
    \centering
    \includegraphics[width=0.9\textwidth]{../Figures/0.050.30.05500-2026-01-26-120745/cluster-patch-patch.png}
\end{figure}


\newpage

\section{Directory: 0.050.30.05500-2026-01-26-134421}

Regrese a las definiciones de~\ref{subsec:tabpot}, pero sin el signo negativo. 

\begin{figure}[h]
    \centering
    \includegraphics[width=0.9\textwidth]{../Figures/0.050.30.05500-2026-01-26-134421/cluster-patch-patch.png}
\end{figure}

No entiendo que está pasando.

\newpage

\section{Acerca del potencial de 3 cuerpos}

Mi intuición con el efecto del potencial está mal.
Según yo, lo que hace el potencial es repeler, es introducir una fuerza que empuja a partículas que están fuera de un enlace entre dos partículas.
Pero no es así.
El potencial se swap reduce el pozo de potencial entre la interacción de patches, permitiendo que sea más fácil romper la interacción.
Voy a probar cambiando el parámetro $w$, para que cuando se haga la suma de fuerzas, no cancelé por completo las interacciones.

Viendo la documentación, hay una forma de guardar el vector de fuerza que se calcúla en cada partícula.
Una forma de debuggear sería guardar esos vectores de fuerza para cada patch cada $N$ pasos y después hacer un script en julia que calculé la fuerza que debería estar asignado a ese patch y contrastar.

\newpage

\section{Directory: 0.050.30.05500-2026-01-27-100615}

Este es el código para evaluar las tablas:
\begin{code}
    th = deg2rad(th);
    r_jk = sqrt(r_ij^2+r_ik^2-2*r_ij*r_ik*cos(th));
 
    f_i1=-w*eps_jk*( DiffU3(eps_ij,eps_ij,sig_p,r_ij) * U3(eps_ik,eps_ik,sig_p,r_ik) );
    f_i2=-w*eps_jk*( U3(eps_ij,eps_ij,sig_p,r_ij) * DiffU3(eps_ik,eps_ik,sig_p,r_ik) );
   
    f_j1=-w*eps_ik*( DiffU3(eps_ij,eps_ij,sig_p,r_ij) * U3(eps_jk,eps_jk,sig_p,r_jk) );
    f_j2=-w*eps_ik*( U3(eps_ij,eps_ij,sig_p,r_ij) * DiffU3(eps_jk,eps_jk,sig_p,r_jk) );

    f_k1=-w*eps_ij*( DiffU3(eps_ik,eps_ik,sig_p,r_ik) * U3(eps_jk,eps_jk,sig_p,r_jk) );
    f_k2=-w*eps_ij*( U3(eps_ik,eps_ik,sig_p,r_ik) * DiffU3(eps_jk,eps_jk,sig_p,r_jk) );
\end{code}

Cambié el parámetro $w$ de \num{1} a \num{0.9}.
No funcionó.

\begin{figure}[h]
    \centering
    \includegraphics[width=0.9\textwidth]{../Figures/0.050.30.05500-2026-01-27-100615/cluster-patch-patch.png}
\end{figure}

Parece que empeoró la situación.
Se redujo la cantidad de enlaces entre dos patches y aumento la cantidad de interacción entre 3 patches.

\newpage

\section{Directory: 0.050.30.05500-2026-01-27-103039}

Se cambió el parámetro $w$ a \num{1.5}.
La temperatura se estabilizo en \num{0.1}, \num{0.05} unidades más de lo esperado.

\begin{figure}[h]
    \centering
    \includegraphics[width=0.9\textwidth]{../Figures/0.050.30.05500-2026-01-27-103039/cluster-patch-patch.png}
\end{figure}

Además quedó eso.
Parece que tienen más energía para estar.

\newpage 

\section{Directory: 0.050.30.05500-2026-01-27-111613}

Cambié de signo, mantuve $w=1.5$.

\begin{figure}[h]
    \centering
    \includegraphics[width=0.9\textwidth]{../Figures/0.050.30.05500-2026-01-27-111613/cluster-patch-patch.png}
\end{figure}

El signo si cambió mucho la configuración de la red.
La temperatura se relajo a \num{0.05}, lo cuál es deseado.

Se va a repetir el experimento con $w=0.9$, considerando que en la situación anterior se tenía elsigno negativo.

\newpage 

\section{Directory: 0.050.30.05500-2026-01-27-114603}

\begin{figure}[h]
    \centering
    \includegraphics[width=0.9\textwidth]{../Figures/0.050.30.05500-2026-01-27-114603/cluster-patch-patch.png}
\end{figure}

Se disminuyo la cantidad de clusters de 5, voy a reducir $w$ a \num{0.5} a ver que pasa.

\newpage

\section{Directory: 0.050.30.05500-2026-01-27-121222}

\begin{figure}[h]
    \centering
    \includegraphics[width=0.9\textwidth]{../Figures/0.050.30.05500-2026-01-27-121222/cluster-patch-patch.png}
\end{figure}

Disminuyeron los clusters de 4.
Los de 3 parece que siguen igual.

Lo bajaré a \num{0.25} a ver que pasa, jaja.

\newpage

\section{Directory: 0.050.30.05500-2026-01-27-123757}

$w=0.25$
\begin{figure}[h]
    \centering
    \includegraphics[width=0.9\textwidth]{../Figures/0.050.30.05500-2026-01-27-123757/cluster-patch-patch.png}
\end{figure}

Pues empeoró.

\newpage

\subsection{Acerca del potencial de 3 cuerpos}

Me dí cuenta que no estaba bien definido el pinche grupo de atomos \emph{PB}.

\begin{code}
# Create patchy particles
group CrossLinker type 1 3
group Monomer type 2 4
group Patches type 3 4
group CL type 1
group MO type 2
group PA type 3
group CM type 1 2
\end{code}

Esto es importante, porque el el potencial de tres cuerpos mapea los tipos de atomos que van a ser sujetos a este potencial a través de la siguiente línea:
\begin{code}
pair_coeff * * threebody/table swapMech.3b NULL NULL PA PB
\end{code}

Esto me indica que puede que el como los patches de los monomeros nunca eran sujetos a este poncial, por eso aún había interacciones de más de 3 patches.

Voy a intentar con un valor de $w=\num{1}$.

\newpage

\section{Directory: 0.050.30.05500-2026-01-27-162300}

\begin{figure}[h]
    \centering
    \includegraphics[width=0.9\textwidth]{../Figures/0.050.30.05500-2026-01-27-162300/cluster-patch-patch.png}
\end{figure}

\begin{figure}[h]
    \centering
    \includegraphics[width=0.9\textwidth]{../Figures/0.050.30.05500-2026-01-27-162300/triple.png}
\end{figure}

Pues no dio lo que quería, pero la distribución si cambió.
El siguiente será con cambio de signo.

\end{document}
